% Options for packages loaded elsewhere
\PassOptionsToPackage{unicode}{hyperref}
\PassOptionsToPackage{hyphens}{url}
\documentclass[
  ignorenonframetext,
  aspectratio=169,
]{beamer}
\newif\ifbibliography
\usepackage{pgfpages}
\setbeamertemplate{caption}[numbered]
\setbeamertemplate{caption label separator}{: }
\setbeamercolor{caption name}{fg=normal text.fg}
\beamertemplatenavigationsymbolsempty
% remove section numbering
\setbeamertemplate{part page}{
  \centering
  \begin{beamercolorbox}[sep=16pt,center]{part title}
    \usebeamerfont{part title}\insertpart\par
  \end{beamercolorbox}
}
\setbeamertemplate{section page}{
  \centering
  \begin{beamercolorbox}[sep=12pt,center]{section title}
    \usebeamerfont{section title}\insertsection\par
  \end{beamercolorbox}
}
\setbeamertemplate{subsection page}{
  \centering
  \begin{beamercolorbox}[sep=8pt,center]{subsection title}
    \usebeamerfont{subsection title}\insertsubsection\par
  \end{beamercolorbox}
}
% Prevent slide breaks in the middle of a paragraph
\widowpenalties 1 10000
\raggedbottom
\AtBeginPart{
  \frame{\partpage}
}
\AtBeginSection{
  \ifbibliography
  \else
    \frame{\sectionpage}
  \fi
}
\AtBeginSubsection{
  \frame{\subsectionpage}
}
\usepackage{iftex}
\ifPDFTeX
  \usepackage[T1]{fontenc}
  \usepackage[utf8]{inputenc}
  \usepackage{textcomp} % provide euro and other symbols
\else % if luatex or xetex
  \usepackage{unicode-math} % this also loads fontspec
  \defaultfontfeatures{Scale=MatchLowercase}
  \defaultfontfeatures[\rmfamily]{Ligatures=TeX,Scale=1}
\fi
\usepackage{lmodern}
\usetheme[]{Boadilla}
\usefonttheme[]{professionalfonts}
\ifPDFTeX\else
  % xetex/luatex font selection
\fi
% Use upquote if available, for straight quotes in verbatim environments
\IfFileExists{upquote.sty}{\usepackage{upquote}}{}
\IfFileExists{microtype.sty}{% use microtype if available
  \usepackage[]{microtype}
  \UseMicrotypeSet[protrusion]{basicmath} % disable protrusion for tt fonts
}{}
\makeatletter
\@ifundefined{KOMAClassName}{% if non-KOMA class
  \IfFileExists{parskip.sty}{%
    \usepackage{parskip}
  }{% else
    \setlength{\parindent}{0pt}
    \setlength{\parskip}{6pt plus 2pt minus 1pt}}
}{% if KOMA class
  \KOMAoptions{parskip=half}}
\makeatother
\setlength{\emergencystretch}{3em} % prevent overfull lines
\providecommand{\tightlist}{%
  \setlength{\itemsep}{0pt}\setlength{\parskip}{0pt}}
\usepackage[]{biblatex}
\addbibresource{bibliography.bib}
\usepackage{bm}
\usepackage{dsfont}
\usepackage{amssymb}
\usepackage{accents}
\usepackage[T2A,T1]{fontenc}


% bold letters (vectors)
\newcommand{\bnull}{\bm{0}}
\newcommand{\ba}{\bm{a}}
\newcommand{\bb}{\bm{b}}
\newcommand{\bc}{\bm{c}}
\newcommand{\bd}{\bm{d}}
\newcommand{\be}{\bm{e}}
% \newcommand{\bf}{\bm{f}}
\newcommand{\bg}{\bm{g}}
\newcommand{\bh}{\bm{h}}
\newcommand{\bi}{\bm{i}}
\newcommand{\bj}{\bm{j}}
\newcommand{\bk}{\bm{k}}
\newcommand{\bl}{\bm{l}}
% \newcommand{\bm}{\bm{m}}
\newcommand{\bn}{\bm{n}}
\newcommand{\bo}{\bm{o}}
\newcommand{\bp}{\bm{p}}
\newcommand{\bq}{\bm{q}}
\newcommand{\br}{\bm{r}}
\newcommand{\bs}{\bm{s}}
\newcommand{\bt}{\bm{t}}
\newcommand{\bu}{\bm{u}}
\newcommand{\bv}{\bm{v}}
\newcommand{\bw}{\bm{w}}
\newcommand{\bx}{\bm{x}}
\newcommand{\by}{\bm{y}}
\newcommand{\bz}{\bm{z}}

% bold letters (matrices)
\newcommand{\bA}{\bm{A}}
\newcommand{\bB}{\bm{B}}
\newcommand{\bC}{\bm{C}}
\newcommand{\bD}{\bm{D}}
\newcommand{\bE}{\bm{E}}
% \newcommand{\bf}{\bm{f}}
\newcommand{\bG}{\bm{G}}
\newcommand{\bH}{\bm{H}}
\newcommand{\bI}{\bm{I}}
\newcommand{\bJ}{\bm{J}}
\newcommand{\bK}{\bm{K}}
\newcommand{\bL}{\bm{L}}
\newcommand{\bM}{\bm{M}}
\newcommand{\bN}{\bm{N}}
\newcommand{\bO}{\bm{O}}
\newcommand{\bP}{\bm{P}}
\newcommand{\bQ}{\bm{Q}}
\newcommand{\bR}{\bm{R}}
\newcommand{\bS}{\bm{S}}
\newcommand{\bT}{\bm{T}}
\newcommand{\bU}{\bm{U}}
\newcommand{\bV}{\bm{V}}
\newcommand{\bW}{\bm{W}}
\newcommand{\bX}{\bm{X}}
\newcommand{\bY}{\bm{Y}}
\newcommand{\hbY}{\hat{\bm{Y}}}
\newcommand{\bZ}{\bm{Z}}


% calligraphic letters
\newcommand{\Acal}{\mathcal{A}}
\newcommand{\Bcal}{\mathcal{B}}
\newcommand{\Ccal}{\mathcal{C}}
\newcommand{\Dcal}{\mathcal{D}}
\newcommand{\Ecal}{\mathcal{E}}
\newcommand{\Fcal}{\mathcal{F}}
\newcommand{\Gcal}{\mathcal{G}}
\newcommand{\Hcal}{\mathcal{H}}
\newcommand{\Ical}{\mathcal{I}}
\newcommand{\Jcal}{\mathcal{J}}
\newcommand{\Kcal}{\mathcal{K}}
\newcommand{\Lcal}{\mathcal{L}}
\newcommand{\Mcal}{\mathcal{M}}
\newcommand{\Ncal}{\mathcal{N}}
\newcommand{\Ocal}{\mathcal{O}}
\newcommand{\Pcal}{\mathcal{P}}
\newcommand{\Qcal}{\mathcal{Q}}
\newcommand{\Rcal}{\mathcal{R}}
\newcommand{\Scal}{\mathcal{S}}
\newcommand{\Tcal}{\mathcal{T}}
\newcommand{\Ucal}{\mathcal{U}}
\newcommand{\Vcal}{\mathcal{V}}
\newcommand{\Wcal}{\mathcal{W}}
\newcommand{\Xcal}{\mathcal{X}}
\newcommand{\Ycal}{\mathcal{Y}}
\newcommand{\Zcal}{\mathcal{Z}}

% greek letters
\newcommand{\eps}{\varepsilon}
\newcommand{\sd}{\sigma}
\newcommand{\ssd}{\sigma^2}
\newcommand{\Sd}{\Sigma}
\newcommand{\Sdi}{\Sigma^{-1}}

\newcommand{\gb}[1]{\beta_#1}
\newcommand{\hbe}[1]{\hat{\beta}_#1}
\newcommand{\beps}{\bm{\varepsilon}}
\newcommand{\hbeps}{\hat{\bm{\varepsilon}}}

\newcommand{\balpha}{\bm{\alpha}}
\newcommand{\bbeta}{\bm{\beta}}
\newcommand{\hbbeta}{\hat{\bm{\beta}}}
\newcommand{\hssd}{\hat{\sigma^2}}
\newcommand{\bchi}{\bm{\chi}}
\newcommand{\bdelta}{\bm{\delta}}
\newcommand{\bepsilon}{\bm{\epsilon}}
\newcommand{\bphi}{\bm{\phi}}
\newcommand{\bgamma}{\bm{\gamma}}
\newcommand{\betah}{\bm{\etah}}
\newcommand{\bpi}{\bm{\pi}}


% prior and posterior parameters
\DeclareSymbolFont{cyrhelper}{T2A}{\familydefault}{m}{n}
\DeclareMathAccent{\post}{\mathord}{cyrhelper}{18}


\newcommand{\btheta}{\bm{\theta}}
\newcommand{\hbtheta}{\hat{\bm{\theta}}}

\newcommand{\thetapri}{\breve{\bm{\theta}}}
\newcommand{\thetapo}{\post{\bm{\theta}}}
\newcommand{\mupri}{\breve{\bm{\mu}}}
\newcommand{\mupo}{\post{\bm{\mu}}}
\newcommand{\Sdpri}{\breve{\Sigma}}
\newcommand{\Sdpo}{\post{\Sigma}}
\newcommand{\Sdipri}{\breve{\Sigma}^{-1}}
\newcommand{\Sdipo}{\post{\Sigma}^{-1}}

\newcommand{\apri}{\breve{a}}
\newcommand{\apo}{\post{a}}
\newcommand{\bpri}{\breve{b}}
\newcommand{\bpo}{\post{b}}

\newcommand{\btaus}{\bm{\tau}^2}
\newcommand{\taus}{\tau^2}


% other
\newcommand{\ty}{\tilde{\bm{y}}}
\newcommand{\tX}{\tilde{\bm{X}}}


\renewcommand{\bar}{\overline}


% statistics
\providecommand{\Pr}{}
\renewcommand{\Pr}{\mathbb{P}}
\newcommand{\Ex}{\mathbb{E}}
\newcommand{\var}{{\mathds{V}\mathrm{ar}}}
\newcommand{\cov}{{\mathds{C}\mathrm{ov}}}
\newcommand{\corr}{{\mathrm{Corr}}}
\newcommand{\ov}{\overline}
\newcommand{\wh}[1]{\widehat{#1}}
\newcommand{\wt}[1]{\widetilde{#1}}
\newcommand{\Cov}{\text{Cov}}
\newcommand{\IG}{\text{IG}}

\newcommand{\sumin}{\sum_{i = 1}^n}
\newcommand{\sumjn}{\sum_{j = 1}^n}
\definecolor{lmugreen}{RGB}{0,136,58}
\setbeamercolor{structure}{fg=lmugreen}
\usecolortheme[named=lmugreen]{structure}
\beamertemplatenavigationsymbolsempty
\usefonttheme{professionalfonts}
\usepackage{listings}
\lstset{
  language=R,
  basicstyle=\scriptsize\ttfamily,
  commentstyle=\ttfamily\color{gray},
  backgroundcolor=\color{white},
  showspaces=false,
  showstringspaces=false,
  showtabs=false,
  tabsize=2,
  captionpos=b,
  breaklines=false,
  breakatwhitespace=false,
  title=\lstname,
  escapeinside={},
  keywordstyle={},
  morekeywords={},
  belowskip=-1.2\baselineskip
}
\usepackage{caption}
\DeclareCaptionFont{tiny}{\tiny}
\captionsetup{font=scriptsize,labelfont=scriptsize,justification=centering}
\usepackage{textpos}
\addtobeamertemplate{frametitle}{}{%
  \begin{textblock*}{100mm}(0.88\textwidth,-0.5cm)
    \includegraphics[height=1cm,width=2cm]{lmu_logo}
  \end{textblock*}}
\AtBeginSection[]{%
  \begin{frame}[noframenumbering,plain]%
    \frametitle{Outline}%
    \setcounter{tocdepth}{1}%
    \tableofcontents[currentsection]%
  \end{frame}}
\AtBeginEnvironment{thebibliography}{\scriptsize}
\usepackage{bookmark}
\IfFileExists{xurl.sty}{\usepackage{xurl}}{} % add URL line breaks if available
\urlstyle{same}
\hypersetup{
  pdftitle={Bayesianische Regression},
  pdfauthor={Lona Koers},
  hidelinks,
  pdfcreator={LaTeX via pandoc}}

\title{Bayesianische Regression}
\subtitle{lineare und logistische Modelle}
\author{Lona Koers}
\date{25. Juli 2025}
\institute{LMU}

\begin{document}
\frame{\titlepage}

\begin{frame}{Motivation und Intuition}
\protect\phantomsection\label{motivation-und-intuition}
TODO: gutes Beispiel

\begin{itemize}
\item
  Generalisierte Lineare Modelle (GLMs)
\item
  Punktvorhersage vs.~Verteilung vorhersagen
\item
  warum reicht uns ein CI / PI
\end{itemize}
\end{frame}

\section{\texorpdfstring{Bayesianische \textbf{lineare}
Modelle}{Bayesianische lineare Modelle}}\label{bayesianische-lineare-modelle}

\begin{frame}{Frequentistisches \(\to\) bayesianisches lineares Modell}
\protect\phantomsection\label{frequentistisches-to-bayesianisches-lineares-modell}
Annahmen:

\begin{enumerate}
\tightlist
\item
  i.i.d. Daten \(\bD = (\by, \bX)\)
\item
  Kondition auf \(\bX\) (implizit)
\end{enumerate}

\textbf{Frequentistisches} lineares Modell:
\(\by \sim \Ncal(\bX \btheta, \ssd \bI)\)

\begin{enumerate}
\setcounter{enumi}{2}
\tightlist
\item
  Gewichtsparameter \(\btheta\) als Zufallsvariable interpretieren
\end{enumerate}

\begin{block}{\textbf{Bayesianisches} lineares Modell:}
\protect\phantomsection\label{bayesianisches-lineares-modell}
\[\by \mid \btheta, \ssd \sim \Ncal(\bX \btheta, \ssd \bI)\]
\end{block}
\end{frame}

\begin{frame}{Modelldefinition (Prior-Verteilungen)}
\protect\phantomsection\label{modelldefinition-prior-verteilungen}
\textbf{Prior}-Annahme für \(\btheta\) (und evtl. \(\ssd\)) notwendig
\(\to\) sehr vielseitige Modell-Anpassung möglich

\begin{block}{1. Normal-Invers-Gamma Prior:}
\protect\phantomsection\label{normal-invers-gamma-prior}
\[
\begin{aligned}
  \btheta \mid \ssd &\sim  \Ncal(\mupri, \ssd \Sdpri) \\
  \ssd &\sim \IG(\apri, \bpri) \\
  \btheta, \ssd &\sim \text{NIG}(\mupri, \ssd \Sdpri, \apri, \bpri)
\end{aligned}
\]

mit Prior Parametern: \(\mupri, \Sdpri, \apri\) und \(\bpri\)
\end{block}

\textbf{Vorteil}: NIG-Prior ist mit Normalverteilungs-Likelihood
konjugiert \(\to\) exakte Inferenz möglich (mehr dazu später)

TODO: Bild
\end{frame}

\begin{frame}{Uninformative Prior als Spezialfall der NIG-Prior}
\protect\phantomsection\label{uninformative-prior-als-spezialfall-der-nig-prior}
\begin{block}{2. Uninformative Prior}
\protect\phantomsection\label{uninformative-prior}
z.B. mit NIG-Prior mit Prior Parametern

\[
\begin{aligned}
\mupri = \bnull&, \quad \Sdipri = \bnull \text{  i.e., } \Sdpri \to \infty \\
\apri = - \frac{p}{2}&, \quad \bpri = 0
\end{aligned}
\]

\(\implies\) flache (und damit uninformative) Prior und maximaler
Einfluss der Daten auf die Posterior:

\[
\btheta \mid \ssd \overset{a}{\sim}  \Ncal(\mupri, \ssd \infty) \; \implies \; p(\btheta\mid \ssd) \propto 1
\]
\end{block}

TODO: Bild
\end{frame}

\begin{frame}{Regularisierung: frequentistisch vs.~bayesianisch}
\protect\phantomsection\label{regularisierung-frequentistisch-vs.-bayesianisch}
\textbf{Erinnerung}: \emph{frequentistische} Regularisierung durch
Minimierung von
\[\text{PLS}(\btheta) = (\by - \bX \btheta)^\top (\by - \bX \btheta) + \lambda \ \text{pen}(\btheta)\]
mit Regularisierungs-Parameter \(\lambda > 0\).

\textbf{Bayesianische Regularisierung} durch Wahl der Prior-Verteilung
für \(\btheta\)
\end{frame}

\begin{frame}{Regularisierung durch Prior Wahl}
\protect\phantomsection\label{regularisierung-durch-prior-wahl}
\begin{block}{3. Ridge Regularisierung}
\protect\phantomsection\label{ridge-regularisierung}
Frequentistisch \autocite{hoerl_ridge_1970,hoerl_ridge_1970-1}:
\(\text{pen}(\btheta) = \|\btheta\|_2^2\)

Bayesianisch \autocite{mackay_bayesian_1992}:
\(\btheta \sim \Ncal(\bnull, \taus \bI)\) mit
\(\taus \propto \frac{1}{\lambda}\)
\end{block}

\begin{block}{4. Lasso Regularisierung}
\protect\phantomsection\label{lasso-regularisierung}
Frequentistisch \autocite{tibshirani_regression_1996}:
\(\text{pen}(\btheta) = \|\btheta\|_1\)

Bayesianisch \autocite{park_bayesian_2008}: \[
\begin{aligned}
\btheta \mid \btaus &\sim \Ncal(\bnull, \btaus \bI) \\
        \taus_j &\overset{\text{i.i.d.}}{\sim} \text{Exp}(0.5 \lambda^2), \quad j = 1, \dots, p
\end{aligned}
\]
\end{block}

Problem: keine Variablenselektion (im Gegensatz zu frequentistischem
Lasso)

\(\to\) Alternative Priors für Variablenselektion: Spike and Slab
\autocite{mitchell_bayesian_1988}, Horseshoe
\autocite{carvalho_horseshoe_2010}, u.v.m.
\end{frame}

\begin{frame}{Regularisierung in Anwendung}
\protect\phantomsection\label{regularisierung-in-anwendung}
Vorteile von bayesianischer Regularisierung sind u.a.:

\begin{itemize}
\tightlist
\item
  Probabilistisches Modell trotz Regularisierung
\item
  Regularisierung-Parameter muss nicht als Hyperparameter optimiert
  werden (z.B. durch Prior auf \(\taus\))
\item
  Mehr Anpassungsmöglichkeiten durch Prior-Spezifikation
\end{itemize}

TODO: Bild regularization Priors + update
\end{frame}

\section{\texorpdfstring{Bayesianische \textbf{generalisierte} lineare
Modelle
(GLMs)}{Bayesianische generalisierte lineare Modelle (GLMs)}}\label{bayesianische-generalisierte-lineare-modelle-glms}

\begin{frame}{Bayesianisches LM \(\to\) \textbf{GLM}}
\protect\phantomsection\label{bayesianisches-lm-to-glm}
\[\text{LM:} \; \by \mid \btheta, \ssd \sim \Ncal(\bX \btheta, \ssd \bI) \quad
\to \quad \text{GLM:} \; \by \mid \btheta \sim F(g^{-1}(\bX \btheta))\]

\begin{itemize}
\tightlist
\item
  Verteilungsannahme von \(\by\) wird (äquivalent zum frequentistischen
  GLM) auf alle Verteilungen \(F\) der Exponentatialfamilie ausgeweitet
\item
  Skala des linearen Prädiktors \(\bX \btheta\) wird mit der
  Link-Funktion \(g^{-1}\) angepasst
\end{itemize}
\end{frame}

\begin{frame}{GLM \(\to\) \textbf{logistisches} Modell}
\protect\phantomsection\label{glm-to-logistisches-modell}
\begin{block}{Bayesianisches logistisches Modell}
\protect\phantomsection\label{bayesianisches-logistisches-modell}
\[
\begin{aligned}
  \by_i \mid \btheta &\sim \text{Bin}(1, g^{-1}(\bx_i \btheta)), \quad i = 1, \dots, n \\
  g^{-1}(\bx_i \btheta) &= \sigma(\bx_i \btheta)
\end{aligned}
\]

Für Beobachtungen \(\bx_i = (1, x_{i1}, \dots, x_{ip})^\top\) und
Sigmoid-Link \(\sigma(y) = \frac{\exp(y)}{1 + \exp(y)}\)
\end{block}

\textbf{Prior Wahl}

\begin{itemize}
\tightlist
\item
  Im Allgemeinen äquivalent zum LM möglich, z.B. Normalverteilung-Prior
\item
  Verteilungen mit schweren Rändern (z.B. t-Verteilung, Cauchy
  Verteilung) verringern Separation und fördern Shrinkage
  \autocite{gelman_weakly_2008,ghosh_use_2017}
\item
  Für Regularisierung können dieselben Priors verwendet werden
  \autocite{ohara_review_2009,fahrmeir_bayesian_2010,van_erp_shrinkage_2019}
\end{itemize}
\end{frame}

\section{Posterior Inference}\label{posterior-inference}

\begin{frame}{Posterior Inference}
Erinnerung: Bayes-Regel
\[p(\btheta \mid \by) = \frac{p(\by \mid \btheta) \; p(\btheta)}{\int p(\by \mid \btheta) \; p(\btheta) d \btheta},\]
wobei \(p(\by \mid \btheta)\) die Modell-Likelihood ist.
\end{frame}

\begin{frame}{bayesianisches LM: exakte Inferenz mit konjugierten
Prioris}
\protect\phantomsection\label{bayesianisches-lm-exakte-inferenz-mit-konjugierten-prioris}
\end{frame}

\begin{frame}{bayesianisches GLM: approximative Inferenz}
\protect\phantomsection\label{bayesianisches-glm-approximative-inferenz}
\begin{block}{Metropolis Hastings (MCMC)}
\protect\phantomsection\label{metropolis-hastings-mcmc}
Idee: Approximation der Posterior mit Markov chain Monte Carlo - was
muss man für Regression anpassen?
\end{block}

\begin{block}{Laplace Approximation (LA)}
\protect\phantomsection\label{laplace-approximation-la}
Idee: Approximation der Posterior mit einer Normalverteilung

\begin{itemize}
\tightlist
\item
  Mittelwert und Varianz werden mit IWLS berechnet
\end{itemize}
\end{block}

\begin{block}{PPD}
\protect\phantomsection\label{ppd}
\end{block}
\end{frame}

\begin{frame}[fragile]{Literatur Empfehlungen}
\protect\phantomsection\label{literatur-empfehlungen}
\begin{itemize}
\tightlist
\item
  Bayesianische Regression (v.a. für praktische Anwendung):
  \textcite{gelman_bayesian_2013}
\item
  Prior Verteilungen (v.a. Shrinkage): \textcite{van_erp_shrinkage_2019}
  und \textcite{celeux_regularization_2012}
\item
  Software: z.B. \texttt{brms} in \texttt{R}, \texttt{PyMC} in
  \texttt{python}
\end{itemize}
\end{frame}

\begin{frame}[allowframebreaks]{Referenzen}
  \bibliographytrue
  \printbibliography[heading=none]
\end{frame}

\end{document}
