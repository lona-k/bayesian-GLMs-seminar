\subsection*{Notation}

We denote prior parameters with $\breve{}$ and posterior parameters with $\post{}$. Vectors are written in bold-face like so $\bx$ and matrices are bold capital letters $\bX$.

\begin{itemize}
    \item[$\btheta$] regression weights
    \item[] 
\end{itemize}



\subsection*{Distributions}

When deriving equations, we assume the following probability density functions and parameter placements:

\begin{itemize}
    \item[$\Ncal(\mu, \ssd)$] Gaussian distribution with mean $\mu$ and variance $\ssd$
    \item[] Gamma distribution
    \item[$IG(a, b)$] Inverse Gamma distribution with scale parameter $a$ and location parameter $b$
    \item[]  (multivariate) Student t-distribution
    \item[] 
\end{itemize}

\subsection*{Proofs and Derivations}

\subsubsection*{Posterior of the Normal-Inverse-Gamma prior}
For the model described in \eqref{eq:NIGprior}, the posterior distribution is calculated according to \citet{fahrmeir_regression_2021} as
\begin{equation*}
    \begin{aligned}
        p(\btheta, \ssd \mid \by) &\overset{\eqref{eq:bayes}}{\propto} \Lcal(\btheta, \ssd \mid \by) p(\btheta, \ssd) \\
        &= \Lcal(\btheta, \ssd \mid \by) p(\btheta \mid \ssd) p(\ssd) \\
        &= \frac{1}{(\ssd)^{n/2}} \exp\bigl( - \frac{1}{2\ssd} (\by - \bX \btheta)^\top(\by - \bX \btheta) \bigr)\\
        &= \frac{1}{(\ssd)^{p/2}} \exp\bigl( - \frac{1}{2\ssd} (\btheta - \mupri)^\top \Sdipri (\btheta - \mupri) \bigr)\\
        &= \frac{1}{(\ssd)^{\apri + 1}} \exp\bigl( - \frac{\bpri}{\ssd} \bigr),
    \end{aligned}
\end{equation*}

which is NIG-distributed

\begin{equation*}
    \btheta, \ssd \mid \by \sim \text{NIG}(\mupo, \Sdpo, \apo, \bpo)
\end{equation*}

with parameters

\begin{equation*}
    \begin{aligned}
        \mupo &= \Sdpo (\Sdipri \mupri + \bX^\top \by) \\
        \Sdpo &= (\bX^\top \bX + \Sdipri)^{-1} \\
        \apo &= \apri + \frac{n}{2}\\
        \bpo &= \bpri + \frac{1}{2} ( \by^\top \by + \mupri^\top \Sdipri \mupo - \mupo^\top \Sdipo \mupo).
    \end{aligned}
\end{equation*}

For the conditional posteriors it holds that
\begin{equation*}
    \begin{aligned}
        \btheta \mid \ssd, \by &\sim \Ncal(\mupo, \ssd \Sdpo) \\
        \btheta \mid \by &\sim \Tcal(2 \apo, \mupo, \bpo / \apo \Sdpo).
    \end{aligned}
\end{equation*}
